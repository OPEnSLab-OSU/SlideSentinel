\documentclass[onecolumn, draftclsnofoot,10pt, compsoc]{IEEEtran}

\usepackage{graphicx}
\usepackage{url}
\usepackage{setspace}

\usepackage{geometry}
\geometry{textheight=9.5in, textwidth=7in}

% 1. Fill in these details
\def \CapstoneTeamName{		        The Slide Sentinel}
\def \CapstoneTeamNumber{		    29}
\def \GroupMemberOne{			    James Stallkamp}
\def \GroupMemberTwo{			    Lucas Campos-Davis}
\def \GroupMemberThree{			    Kevin Koos}
\def \CapstoneProjectName{		    Slide Sentinel}
\def \CapstoneSponsorCompany{	    Oregon State University}
\def \CapstoneSponsorPerson{		Dr Chet Udell}

% 2. Uncomment the appropriate line below so that the document type works
\def \DocType{	Problem Statement
				%Requirements Document
				%Technology Review
				%Design Document
				%Progress Report
				}
			
\newcommand{\NameSigPair}[1]{\par
\makebox[2.75in][r]{#1} \hfil 	\makebox[3.25in]{\makebox[2.25in]{\hrulefill} \hfill		\makebox[.75in]{\hrulefill}}
\par\vspace{-12pt} \textit{\tiny\noindent
\makebox[2.75in]{} \hfil		\makebox[3.25in]{\makebox[2.25in][r]{Signature} \hfill	\makebox[.75in][r]{Date}}}}
% 3. If the document is not to be signed, uncomment the RENEWcommand below
\renewcommand{\NameSigPair}[1]{#1}

\bibliographystyle{IEEEtran}


%%%%%%%%%%%%%%%%%%%%%%%%%%%%%%%%%%%%%%%
\begin{document}
\begin{titlepage}
    \pagenumbering{gobble}
    \begin{singlespace}
        \hfill 
        % 4. If you have a logo, use this includegraphics command to put it on the coversheet.
        %\includegraphics[height=4cm]{CompanyLogo}   
        \par\vspace{.2in}
        \centering
        \scshape{
            \huge CS Capstone \DocType \par
            {\large\today}\par
            \vspace{.5in}
            \textbf{\Huge\CapstoneProjectName}\par
            \vfill
            {\large Prepared for}\par
            \Huge \CapstoneSponsorCompany\par
            \vspace{5pt}
            {\Large\NameSigPair{\CapstoneSponsorPerson}\par}
            {\large Prepared by }\par
            Group\CapstoneTeamNumber\par
            % 5. comment out the line below this one if you do not wish to name your team
            \CapstoneTeamName\par 
            \vspace{5pt}
            {\Large
                \NameSigPair{\GroupMemberOne}\par
                \NameSigPair{\GroupMemberTwo}\par
                \NameSigPair{\GroupMemberThree}\par
            }
            \vspace{20pt}
        }
        \begin{abstract}
        	Landslides are natural disasters which are extremely costly to clean up and cause serious damage to infrastructure. These landslides are not only costly in their clean-up, but also cost organizations much in loss of operations due to delays. Large land owners such as Weyerhauser are at particular risk to delays in logging operations due to land slides being found long after they have occurred and operations are in motion. Our goal is to develop a landslide sensor detection network which can be utilized to track different possible sites of landslides for shifts in the landscape. To achieve this, Arduino sensors will be developed to track and relay information back to a central hub where the data can be shown visually in an online client. The product will be tested for use in the outdoors. The finished product will consist of a set of design documents for the sensor and the source code for the sensor, server application, and online client.
        \end{abstract}     
    \end{singlespace}
\end{titlepage}



\newpage
\pagenumbering{arabic}
\tableofcontents
% 7. uncomment this (if applicable). Consider adding a page break.
%\listoffigures
%\listoftables
\clearpage

% 8. now you write!
\section{Description and Definitions}
Mass flows of soil, otherwise known as landslides, are classified as a type of mass wasting where a large mass flows down a downward slope. Landslides develop overtime due to an accumulation of water in the soil and a lack of support structures in the ground. These landslides occur naturally along slopes but changes in the local geography, such as road construction, can weaken slopes and increase the likelihood of a landslide.\cite{glade2003landslide} These landslides occur abruptly and quickly without warning so care must be taken to minimize their likelihood and track possible sites. While a landslide might start out small, as they flow down the slope they can gain mass and travel farther, further picking up more material. Landslides will even carry large rocks and boulders further compounding possible damages. Due to the sheer size of these landslides, clean-up and repair of the natural disaster sites are a costly endeavor. According to the U.S.G.S., in 2005 they estimated that landslides cost the United States around \$3.5 billion dollars per year just in repair damage.\cite{dangerlandslide} Timberland owners, such as Weyerhauser, are particularly concerned about landslides due to the possible damages they can cause to logging roads and the interruptions to logging operations. Due to their rapidly growing population and often difficult terrain, third world countries are poorly equipped to deal with these natural disasters.\cite{guzzetti1999landslide} If these landslides could be tracked, organizations can more readily respond and handle these situations minimizing potential losses in operations. The goal of the Slide Sentinel project is to develop a sensor network which can detect shifts in position and relay these changes back to a central data hub where the data can be analyzed and presented. The sensors will collect data periodically on position and orientation using an accelerometer and GPS. This data will be collected and presented in an online client on Google maps, tracking the sensor node statuses and their individual movements. We aim for this project to be easily deployable in a variety of outdoor conditions so as to help land managers better understand and track landslides on their property.

\section{Proposed Solution}
Slide Sentinel is a wireless network of sensors that monitors for positional shifts. Slide Sentinel will have remote senors that can detect and transmit data on position, orientation, and acceleration. The data that Slide Sentinel collects will be stored in a data base and made access able through an online client. Slide Sentinel is made up of three distinct components, a sensor module, a central hub, and an online client.

\subsection{Sensor Module}
The module will consist of a 3D-printed enclosure for a small Arduino, a battery, and any additional sensors needed for function. A prototype sensor will be provided by our client which is capable of sensing shifts in movement and sending a signal. These sensors will track their position through GPS and send a signal to the central hub through radio letting it know of its current position, orientation, and time. 

\subsection{Central Hub}
The central hub will also consist of a 3D-printed enclosure. It will contain a small board, a battery, a receiver module to get data from the sensor array, a 4G LTE module to send data to the online client, and possibly a solar panel to recharge its battery pack. It will take data from the sensor modules and transmit it to the online client periodically.
\subsection{Online client}
The Online Client is a web page that provides access to the data collected by slide Sentinel. This web page will be able to display raw data as well as a visualization of the data and the location it came from. This visualization will appear as a map of the specific area with data points projected onto the map to show where and by how much any shifts occurred. 

\section{Performance Metrics}
Performance metrics are broken up into three sections corresponding to the three parts of the project.
\subsection{Sensor Module}
\begin{itemize}
    \item Durability
        \begin{itemize}
            \item The sensor will be located in outdoor conditions where it will be subject to water, wind, and various other surface conditions. The sensor will have to weather a landslide so the case must also be able withstand impacts that could happen during one.
        \end{itemize}
    \item Battery Life
    \begin{itemize}
            \item Sensors will need to be deployed in the field for a length of time before needing any kind of services. This battery lifetime should be well understood so routine maintenance can be made. The sensor software will need to be efficient enough to sustain continuous use for a period of time. 
        \end{itemize}
\end{itemize}
\subsection{Central Hub}
We will need to evaluate the central hub on three characteristics: power, durability, and communication with the online client.
\begin{itemize}
    \item Power
        \begin{itemize}
            \item The central hub either needs to be able to power itself or have a battery pack that can power it long enough to maximize the time between battery pack changes.
        \end{itemize}
    \item Durability
        \begin{itemize}
            \item We assume that the hub will also be exposed to the same conditions that the sensor modules. It will need to be sealed against the elements. The central hub will also need to be able to survive any potential impacts that may happen.
        \end{itemize}
    \item Communication
        \begin{itemize}
            \item The central hub will need to be able to receive communications from the sensor nodes and have enough memory to store the data received from them. The central hub will need to be able to consistently connect to the online client even in weather conditions that could lead to a landslide.
        \end{itemize}
\end{itemize}

\subsection{Online Client}
The online client will be evaluated on three primary characteristics: maintainability, portability, and usability.
\begin{itemize}
    \item Maintainability
        \begin{itemize}
            \item The online client must be easy to maintain and expand.
            The online client will be accessed remotely and must be easy to maintain or fix when necessary.
            Also in the future it is likely more functionality will be added to the online client, meaning the online client must be easy to modify or expand.
        \end{itemize}
    \item Portability
        \begin{itemize}
            \item The online client can be accessed through the internet, as such the online client needs to be portable to most modern web browsers. This means that the online client should both function and appear visually the same on any modern web browser.
        \end{itemize}
    \item Usability
        \begin{itemize}
            \item The online client could be accessed by a large variety of uses and therefore should be simple to use in order to accommodate that variety. Users should be able to easy and intuitively access the web page, retrieve data, and visualize data.
        \end{itemize}
\end{itemize}
\newpage
\bibliography{bibliography}
% forces full bib to show
%\nocite{*}
\end{document}